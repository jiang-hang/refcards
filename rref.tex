\documentclass[10pt,twocolumn]{article}


\usepackage{xeCJK}
\setCJKmainfont[BoldFont={SimHei},ItalicFont={KaiTi}]{FangSong}
\setCJKsansfont{SimHei}
\setCJKmonofont{FangSong}

\usepackage{graphicx}
\usepackage{epstopdf}
\usepackage{alltt}
\usepackage{amssymb}
\usepackage{xcolor}
\usepackage{hyperref}

\hypersetup{colorlinks=true}

\textheight = 10in
\oddsidemargin = 0.0 in
\evensidemargin = 0.0 in
\topmargin = 0.0 in
\headheight = 0.0 in
\headsep = 0.0 in
\parskip = 1pt
\parindent = 0.0in

\setlength{\hoffset}{-16pt}
\setlength{\voffset}{-26pt}

\newlength{\niem}
\setlength{\niem}{-1.2em}
\newlength{\niiem}
\setlength{\niiem}{-1.5em}

\newcommand{\ttnrbf}[1]{\textnormal{\textbf{#1}}}

\newcommand{\ttitbf}[1]{\textnormal{\textbf{#1}}}

\newcommand{\ttit}[1]{\textit{#1}}
\newcommand{\sSecBox}[1]{\begin{minipage}{3in}\begin{alltt}{\small\textrm{\textbf{#1}}}}
\newcommand{\eSecBox}{\end{alltt}\end{minipage}\\ \rule{0pt}{0pt}\hrule}

\newcommand{\sSecMBox}[1]{\begin{minipage}{3in}{\small\textrm{\textbf{#1}}}\\}
\newcommand{\eSecMBox}{\end{minipage}\\ \rule{0pt}{0pt}\hrule}


\newcommand{\srb}[2]{\parbox[t]{#1}{#2}\rule{5pt}{0pt}}

\newcommand{\blt}{\makebox[2ex]{\(\bullet\)}}
\newcommand{\bsp}{\makebox[2ex]{}}
\newcommand{\bks}{\(\backslash\)}

\newcommand{\meo}[1]{\mbox{\textcolor{red}{#1}}}
\newcommand{\spl}[1]{\mbox{\textcolor{blue}{#1}}}

\begin{document}
\pagestyle{empty}

\fbox{{\parbox[b]{3in}{\huge \bf R ref.\\ \tiny 杨学辉 \url{http://www.bagualu.net}  \\ 最近更新 \today}}}
\tiny

\sSecBox{获得帮助}
\srb{55pt}{\blt 查看可用的包        }\srb{45pt}{library()    }\srb{55pt}{\blt 查看某个包的信息}\srb{45pt}{help(package="xx")}
\srb{55pt}{\blt 查看已装载的包    }\srb{45pt}{search()  }\srb{55pt}{\blt 函数的帮助     }\srb{45pt}{help(mean) | ?mean}
\srb{55pt}{\blt 所有包含mean的函数}\srb{45pt}{apropos("mean")  }\srb{55pt}{\blt 所有的类型名}\srb{45pt}{methods("is")}    
\eSecBox

\sSecBox{常用操作}
\srb{55pt}{\blt 获得当前工作目录}\srb{45pt}{getwd()          }\srb{55pt}{\blt 设置工作目录     }\srb{45pt}{setwd("../") }
\srb{55pt}{\blt 列目录        }\srb{45pt}{dir(path,pattern)}\srb{55pt}{\blt 列出所有方法a    }\srb{45pt}{methods(a)    }
\eSecBox

\sSecBox{读取数据}
\srb{55pt}{\blt 读取数据框        }\srb{45pt}{read.table}\srb{55pt}{\blt 读取csv    }\srb{45pt}{read.csv    }
\srb{55pt}{\blt 用ff包分块读取csv        }\srb{45pt}{ff::read.csv.ffdf("aa.csv",head=T,first.rows=10,next.rows=20)}
\eSecBox

\sSecBox{变量信息}
\srb{50pt}{\blt is.na(x)}\srb{50pt}{\blt is.null(x) }\srb{50pt}{\blt is.array(x)}\srb{50pt}{\blt is.data.frame(x) }
\srb{50pt}{\blt is.numeric(x)}\srb{50pt}{\blt is.complex(x) }\srb{50pt}{\blt is.character(x)}
\srb{55pt}{\blt 向量长度        }\srb{45pt}{length(x)}\srb{55pt}{\blt 行数    }\srb{45pt}{nrow(x)    }
\srb{55pt}{\blt 类名        }\srb{45pt}{class(x)}\srb{55pt}{\blt 列数    }\srb{45pt}{ncol(x)    }
\eSecBox

\sSecBox{数据选择及操作}
\srb{55pt}{\blt 向量的某些元素        }\srb{45pt}{x[c(1,2,6)]}\srb{55pt}{\blt 向量去除某些元素        }\srb{45pt}{x[-c(1,3,6)]}
\srb{55pt}{\blt 向量去除某些元素        }\srb{45pt}{x[c(-1,-2,-6)]}\srb{55pt}{\blt 向量大于3的元素        }\srb{45pt}{x[x>3]}
\srb{55pt}{\blt 列表转为向量        }\srb{45pt}{unlist(x)}\srb{55pt}{\blt x的倒序        }\srb{45pt}{rev(x)}
\srb{55pt}{\blt 排序        }\srb{45pt}{sort(x)}\srb{55pt}{\blt x的最大值索引        }\srb{45pt}{which.max(x)}
\srb{55pt}{\blt 分段x构造hist        }\srb{45pt}{cut(x,breaks)}\srb{55pt}{\blt x的最小值索引        }\srb{45pt}{which.min(x)}
\eSecBox

\sSecBox{绘图}
\srb{55pt}{\blt 饼图        }\srb{45pt}{pie(c(10,2,3),labels=letters[1:3])}
\srb{55pt}{\blt 柱状图        }\srb{45pt}{hist(c(1,2,3,4,56))}
\srb{55pt}{\blt 折线图        }\srb{45pt}{plot(c(1,2,3,4,56),type='l')}
\srb{55pt}{\blt 函数图        }\srb{45pt}{plot(sin,-3,3)}
\srb{55pt}{\blt 自定义函数图  }\srb{45pt}{plot(function(x) 3*x*x+5*x ,-3,3)}
\srb{55pt}{\blt 绘制多图      }\srb{45pt}{par(mfrow=c(2,3)) 然后plot...} 
\eSecBox


\end{document}
